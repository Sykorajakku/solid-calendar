\chapter{Analysis}

\section{Calendar Features}

\subsection*{Calendar Events}

\subsection*{Calendar Views}
The application should support displaying calendar events in different calendar views, such as a view of calendar events in a single day or week. Views in different timespans increase the application's usability by hiding non-important events or providing a valuable overview of future events.
In the following subsection, we will introduce functional and non-functional requirements that each calendar view should support and then define specific calendar views that should be available in the application.

\subsubsection*{Shared functionality of calendar views}
In this subsection, we define a set of features that each calendar view should offer in the application. Defining such features increases the application's usability by defining expectable behavior shared in each calendar view.
As we want to make the calendar application interactive, users should be able to modify calendar events displayed in a calendar view by simply moving the visualization of the event from one timeslot to another. By modifying an event in a calendar view, users can set a different start time for an event or change the duration of an event, as shown in figure [X], where the user \textit{drag and drops} an event to a different location, changing start of the event.

\begin{enumerate}[label=\color{reqcolor}\textbf{R{\arabic*}}]
    \item \label{app:req:dragdrop} Users should be able to interactively modify start and duration of events displayed in calendar views.
\end{enumerate}

\subsubsection*{Day view}

\subsubsection*{Week view}

\subsubsection*{Month view}

\subsection*{Sharing}

\subsection*{Import or Export of iCalendar Format}

\section{SOLID features}
